\newpage
\section*{4. L\'{o}gica Combinacional}
Un circuito consiste en variables de entrada, compuertas l\'{o}gicas y
variables de salida. Las $n$ variables de entrada provienen de una fuente externa,
las $m$ variables de salidas van a un destino externo. Si se incluyen registros de
almacenamiento entonces se considera un circuito secuencial.
Con $n$ variables de entrada hay $2^n$ posibles combinaciones de entradas binarias.
Para cada una hay un posible valor de salida. Por tanto es posible especificar una
tabla de verdad con los valores de salida para cada combinaci\'{o}n de entradas.

Se presentar\'{a} circuitos combinacionales est\'{a}ndar m\'{a}s importantes, como
los sumadores, restadores, comparadores, decodificadores, codificadores, multiplexores,
desmultiplexores.
Estos componentes se fabrican como circuitos MSI (Medium Scale Integration) y tambi\'{e}n
se usan como celdas est\'{a}ndar en circuitos VSLI complejos tanto como ASIC.

\subsection*{4.1. Procedimiento de an\'{a}lisis}
Se refiere a deducir la funci\'{o}n l\'{o}gica que realiza el circuito. Este proceso
parte de un diagrama l\'{o}gico dado y culmina en conjunto de funciones booleanas, una
tabla de verdad o una posible explicaci\'{o}n del funcionamiento del circuito.
\medbreak

\begin{itemize}
    \item
          El primer paso es asegurarse que el circuito sea combinacional y no secuencial. Lo cual
          se puede asegurar si no hay elementos de almacenamiento o trayectorias de retroalimentaci\'{o}n.
    \item
          Una vez asegurado se procede a obtener las funciones booleanas de salida o la tabla de verdad.
\end{itemize}

\subsubsection*{4.1.1 Pasos para obtener las funciones booleanas de salida}
\begin{enumerate}
    \item Rotular con s\'{i}mbolos arbitrarios todas las salidas de compuerta que son
          funci\'{o}n de variables de entrada. Determinar esto para cada salida de compuerta.
    \item Rotular con s\'{i}mbolos arbitrarios las compuertas que son funci\'{o}n de
          variables de entrada y de compuertas previamente rotuladas.
    \item Repetir el paso 2 hasta obtener las salidas del circuito.
    \item Obtener las funciones booleanas de salida en t\'{e}rminos de las variables de entrada.
\end{enumerate}

\subsubsection*{4.1.2 Deducci\'{o}n de la tabla de verdad}
\begin{enumerate}
    \item Determinar el n\'{u}mero de variables de entrada del circuito, paran $n$ entradas,
          forme $2^n$ posibles combinaciones y bosqueje una lista de $0$ a $2^n - 1$.
    \item Rotular las salidas de las compuertas selectas con s\'{i}mbolos arbitrarios.
    \item Obtener la tabla de verdad para las salidas de las compuertas que son funci\'{o}n
          \'{u}nicamente de las variables de entrada.
    \item Obtener la tabla de verdad para las salidas de las compuertas que son funci\'{o}n
          de valores previamente definidos.
\end{enumerate}

\subsection*{4.2. Procedimiento de dise\~{n}o}
Parte de la especificaci\'{o}n del problema y culmina en un diagrama l\'{o}gico de circuitos
o un conjunto de funciones booleanas a partir de las cuales se puede obtener un diagrama l\'{o}gico.

\subsubsection*{4.2.1. Pasos para obtener el diagrama l\'{o}gico}
\begin{enumerate}
    \item Deducir el n\'{u}mero requerido de entradas y salidas; asignar s\'{i}mbolos a cada una.
    \item Deducir la tabla de verdad que define la relaci\'{o}n  entre las entradas y las salidas.
    \item Obtener las funciones booleanas simplificadas para cada salida en funci\'{o}n de las entradas.
    \item Dibujar el diagrama l\'{o}gico y verificar que el dise\~{n}o sea correcto.
\end{enumerate}

La tabla de verdad de un circuito combinacional consta de columnas de entrada y columnas de salida.
Las columnas de entrada se obtinen de los $2^n$ n\'{u}meros binarios para las $n$ variables de entrada.
Los valores binarios de salida se deducen de las especificaciones planteadas. Tales especificaciones
suelen ser incompletas, y cualquier interpretaci\'{o}n err\'{o}nea podr\'{i}a dar pie a una tabla de verdad
incorrecta.

Las funciones de salida se simplifican con cualquier m\'{e}todo disponible, as\'{i} como mapas de karnaugh,
manipulaci\'{o}n algebraica, o un programa de computadora.

\subsection*{4.3. Sumador-restador binario}
Es un circuito combinacional que realiza operaciones aritm\'{e}ticas de suma y resta con n\'{u}meros
binarios.
\newpage

\subsubsection*{4.3.1 Semisumador}
Necesita dos entradas binarias y dos salidas binarias. Las variables de entrada designan los bits sumandos;
las salidas, la suma y el acarreo. Las funciones booleanas simplificadas para las dos salidas se obtienen
directamente de la tabla de verdad. Estas son:
\begin{align*}
    S & = x'y + xy' \rightarrow x \oplus y \\
    C & = xy
\end{align*}

\begin{flushleft}
    La tabla de verdad ser\'{i}a:
\end{flushleft}

\begin{table}[h]
    \centering
    \begin{tabular}{cc|cc}
        \toprule
        $x$ & $y$ & $C$ & $S$ \\
        \midrule
        0   & 0   & 0   & 0   \\
        0   & 1   & 0   & 1   \\
        1   & 0   & 0   & 1   \\
        1   & 1   & 1   & 0   \\
        \bottomrule
    \end{tabular}
    \caption{Semisumador}
\end{table}

\subsubsection*{4.3.2 Sumador completo}
Necesita tres entradas binarias y dos salidas binarias. Las variables de entrada designan los bits sumandos
y el acarreo de entrada; las salidas, la suma y el acarreo de salida. Las funciones booleanas simplificadas
para las dos salidas se obtienen directamente de la tabla de verdad. Estas son:
\begin{align*}
    S & = x'y'z + xyz' + xy'z' + xyz \rightarrow x \oplus y \oplus z \\
    C & = xy + xz + yz
\end{align*}

\begin{flushleft}
    La tabla de verdad ser\'{i}a:
\end{flushleft}

\begin{table}[h]
    \centering
    \begin{tabular}{ccc|cc}
        \toprule
        $x$ & $y$ & $z$ & $C$ & $S$ \\
        \midrule
        0   & 0   & 0   & 0   & 0   \\
        0   & 0   & 1   & 0   & 1   \\
        0   & 1   & 0   & 0   & 1   \\
        0   & 1   & 1   & 1   & 0   \\
        1   & 0   & 0   & 0   & 1   \\
        1   & 0   & 1   & 1   & 0   \\
        1   & 1   & 0   & 1   & 0   \\
        1   & 1   & 1   & 1   & 1   \\
        \bottomrule
    \end{tabular}
    \caption{Sumador completo}
\end{table}

\newpage
\subsubsection*{4.3.3 Sumador binario}
Produce la suma aritm\'{e}tica de dos n\'{u}meros binarios. Es posible construirlo con
sumadores completos dispuestos en cascada, conectando el acarreo de salida de un sumador
completo al acarreo de entrada del siguiente.

\subsubsection*{4.3.4 Restador binario}
La forma m\'{a}s conveniente de efectuar la resta de n\'{u}meros binarios sin signo es
utilizando complementos. La resta de $A - B$ se efect\'{u}a obteniendo el complemento
a dos de $B$ y sum\'{a}ndolo a $A$. El complemento a dos de un n\'{u}mero binario se
obtiene calculando el complemento a uno y sum\'{a}ndole 1 al par de bits menos significativo.
El complemento a uno se implementa con inversores y el 1 se suma a trav\'{e}s de un acarreo
de entrada.

Las operaciones de suma y resta se pueden combinar en un solo circuito que tiene un sumador
binario compartido. Esto se realiza con la inclusi\'{o}n de una compuerta XOR con cada sumador
completo. Agregando una entrada de control $M$ para decidir si se efect\'{u}a una suma o una
resta, se obtiene un sumador-restador binario.

\subsubsection*{4.3.5 Sumador decimal}
Un sumador decimal requiere como m\'{i}nimo nueve entradas y cinco salidas, ya que se requieren
cuatro bits para codificar cada d\'{i}gito decimal y el circuito necesita un acarreo de entrada y
uno de salida. Se puede implementar utilizando el c\'{o}digo BCD.

\subsection*{4.4. Multiplicador binario}
La multiplicaci\'{o}n de n\'{u}meros binarios se efect\'{u}a igual que la de n\'{u}meros decimales.
El multiplicando se multiplica por cada bit del multiplicador, comenzando por el menos significativo.
Cada una de estas multiplicaciones forma un producto parcial. Los productos parciales se suman para
obtener el producto final.
\begin{flushleft}
    Para construir un multiplicador se requiere:
\end{flushleft}
\begin{itemize}
    \item Multiplicador de $J$ bits.
    \item Multiplicando de $K$ bits.
    \item $(J \times K)$ compuertas AND.
    \item $(J - 1)$ sumadores de $K$ bits.
\end{itemize}
Se obtiene un producto de $(J + K)$ bits.

\subsection*{4.5 Comparador de magnitudes}
La comparaci\'{o}n de dos n\'{u}meros es una operaci\'{o}n que determina si un n\'{u}mero es mayor que,
menor que o igual a otro n\'{u}mero. \textit{Un comparador de magnitudes} es un circuito combinacional
que compara dos n\'{u}meros, $A$ y $B$, y determina sus magnitudes relativas. El resultado de la comparaci\'{o}n
se especifica con tres variables binarias que indican si $A > B \vee A = B \vee A < B$.

\begin{flushleft}
    Si queremos comparar los siguientes dos n\'{u}meros binarios:
\end{flushleft}
\begin{center}
    $A = A_3A_2A_1A_0$ \\
    $B = B_3B_2B_1B_0$
\end{center}
\begin{flushleft}
    Cada letra con sub\'{i}ndice representa uno de los d\'{i}gitos del n\'{u}mero. Los dos n\'{u}meros son
    iguales si todos los pares de d\'{i}gitos significativos son iguales.
\end{flushleft}
\begin{center}
    $A_3 = B_3$ \\
    $A_2 = B_2$ \\
    $A_1 = B_1$ \\
    $A_0 = B_0$
\end{center}
\begin{flushleft}
    Se puede implementar con una funci\'{o}n XNOR as\'{i}:
\end{flushleft}
\begin{center}
    $x_i = A_iB_i + A_i'B_i'$ \quad para $i = 0, 1, 2, 3$
\end{center}
\begin{flushleft}
    donde $x_i = 1$ \'{u}nicamente si los dos bits de la posici\'{o}n $i$ son iguales.
    Para que exista la condici\'{o}n de igualdad, las $x_i$ deben ser todas 1. Lo que implica una
    operaci\'{o}n AND de todas las variables.
\end{flushleft}
\begin{center}
    $(A = B) = x_3x_2x_1x_0$
\end{center}
\begin{flushleft}
    Para determinar si $A > B \vee A < B$, se inspeccionan las magnitudes relativas de pares de d\'{i}gitos
    significativos, comenzando por el m\'{a}s significativo. Si los dos d\'{i}gitos son iguales, se comparar\'{a}
    el siguiente par de d\'{i}gitos menos significativos; as\'{i} sucesivamente hasta encontrar un par de d\'{i}gitos
    distinto.
\end{flushleft}
\begin{center}
    Si $A = 1 \wedge B = 0$ \quad entonces \quad $A > B$, \\
    Si $A = 0 \wedge B = 1$ \quad entonces \quad $A < B$
\end{center}
\begin{flushleft}
    Esta comparaci\'{o}n sucesiva se expresa l\'{o}gicamente con las dos funciones booleanas:
\end{flushleft}
\begin{center}
    $(A > B) = A_3B_3' + x_3A_2B_2' + x_3x_2A_1B_1' + x_3x_2x_1A_0B_0'$, \\
    $(A < B) = A_3'B_3 + x_3A_2'B_2 + x_3x_2A_1'B_1 + x_3x_2x_1A_0'B_0$
\end{center}
\begin{flushleft}
    Los s\'{i}mbolos $(A > B) \wedge (A < B)$ son variables binarias de salida que valen 1 cuando $A > B$ y $A < B$,
    respectivamente.
\end{flushleft}

\subsection*{4.6. Decodificadores}
En los sistemas digitales, las cantidades discretas de informaci\'{o}n se representan con c\'{o}digos binarios.
Un c\'{o}digo binario de $n$ bits puede representar hasta $2^n$ cantidades distintas.
\begin{center}
    Un \textit{decodificador} es un circuito combinacional que convierte la informaci\'{o}n binaria de \textbf{$n$ l\'{i}neas
    de entrada} a un m\'{a}ximo de \textbf{$2^n$ l\'{i}neas de salida.}
\end{center}
\begin{flushleft}
    Aunque podr\'{i}a tener un n\'{u}mero menor de l\'{i}neas de salida, as\'{i}: \\
    \begin{center}
        $(dec): n$ a $m$ l\'{i}neas de salida, donde $\quad m \leq 2^n$
    \end{center}
\end{flushleft}

Un decodificador podr\'{i}a operar con salidas complementadas o no complementadas. Podr\'{i}a tener
una entrada de habilitaci\'{o}n que debe satisfacer una condici\'{o}n l\'{o}gica dada para habilitar
el circuito. Un \textit{decodificador} con entrada de habilitaci\'{o}n puede funcionar como \textbf{desmultiplexor}.
Un \textit{desmultiplexor} es un circuito que recibe informaci\'{o}n de una sola l\'{i}nea y la dirige a una
de $2^n$ l\'{i}neas de salida.

Dado que se obtienen operaciones de decodificador y desmultiplexor con el mismo circuito, decimos que un
decodificador con entrada de habilitaci\'{o}n es un \textit{decodificador/desmultiplexor}.
Es posible conectar los decodificadores con entradas de habilitaci\'{o}n unos con otros para formar un circuito
decodificador m\'{a}s grande.

\subsection*{4.7. Codificadores}

\subsection*{4.8. Multiplexores}

\subsection*{4.9. Demultiplexores}