\newpage
\section*{4. L\'{o}gica Combinacional}
Los circuitos l\'{o}gicos para
sistemas digitales pueden ser combinacionales o secuenciales. Un circuito
combinacional consiste en compuertas l\'{o}gicas cuyas salidas en cualquier
momento est\'{a}n determinadas por la combinaci\'{o}n actual de entradas. Por
otra parte los circuitos secuenciales usan elementos de almacenamiento
adem\'{a}s de compuertas l\'{o}gicas, y sus salidas son funci\'{o}n de las
entradas y del estado de los elementos de almacenamiento. Esto a su vez es
funci\'{o}n de entradas anteriores. Por ello las salidas de un circuito
secuencial dependen no s\'{o}lo de los valores actuales de las entradas, sino
tambi\'{e}n de las entradas anteriores, adem\'{a}s el comportamiento del
circuito se debe especificar con una sucesi\'{o}n temporal de entradas y estados
internos.

