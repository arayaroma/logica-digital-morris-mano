\newpage
\section*{3. Minimizaci\'{o}n en el nivel de compuertas}
\subsection*{3.1 El m\'{e}todo del mapa}
La complejidad de las compuertas de l\'{o}gica digital que implementan una
funci\'{o}n booleana est\'{a} relacionada directamente con la complejidad de la
expresi\'{o}n algebraica de la funci\'{o}n. aunque la representaci\'{o}n como
una tabla de verdad es \'{u}nica, hay muchas maneras de expresarla
algebraicamente.

El m\'{e}todo del mapa ofrece un procedimiento para simplificar funciones
booleanas. Se podr\'{i}a considerar como una versi\'{o}n pict\'{o}rica de una
tabla de verdad. Se conoce como mapa de \textbf{Karnaugh} o mapa K.

Es un diagrama hehco de cuadrados, cada uno representando un minit\'{e}rmino de
la funci\'{o}n. Al reconocer diversos patrones, el usuario puede deducir
expresiones algebraicas alternas para la misma funci\'{o}n y luego escoger la
m\'{a}s simple. Las expresiones est\'{a}n expresadas en una de las dos formas
est\'{a}ndar, suma de productos o producto de sumas. Esto produce un diagrama de
circuito con el m\'{i}mino de compuertas y el m\'{i}mino de entradas a cada
compuerta. La expresi\'{o}n m\'{a}s simple no es \'{u}nica.

\subsubsection*{3.1.1 Mapas de dos variables}
Para un mapa de dos variables, hay cuatro minit\'{e}rminos $m_0, m_1, m_2, m_3$,
por tanto el mapa consiste en cuatro cuadrados, uno para cada minit\'{e}rmino.
Si se marca los cuadrados cuyos minit\'{e}rminos pertenecen a una funci\'{o}n dada,
el mapa se convertir\'{a} en otra forma \'{u}til de representar cualquiera de las 16
funciones booleanas de dos variables.

\subsubsection*{3.1.2 Mapas de tres variables}
En un mapa de tres variables hay ocho minit\'{e}rminos para tres variables binarias;
por tanto el mapa consta de ocho cuadrados.
\begin{center}
    $m_0 = 000, m_1 = 001, m_2 = 011, m_3 = 010$ \\
    $m_4 = 110, m_5 = 111, m_6 = 101, m_7 = 100$
\end{center}

Advertir que los minit\'{e}rminos no est\'{a}n acomodados en sucesi\'{o}n binaria, sino en
c\'{o}digo Gray. La caracter\'{i}stica es que s\'{o}lo un bit cambia de valor
entre dos columnas adyacentes.

El n\'{u}mero de cuadrados adyacentes que es posible combinar \textbf{siempre}
debe ser una potencia de 2, como 1, 2, 4, 8. Al aumentar el n\'{u}mero de
cuadrados adyacentes que se combinan, se reduce el n\'{u}mero de literales del
t\'{e}rmino producto obtenido.
\begin{itemize}
    \item Un cuadrado representa un minit\'{e}rmino $\rightarrow$ t\'{e}rmino con tres literales.
    \item Dos cuadrados adyacentes representan dos minit\'{e}rminos $\rightarrow$ t\'{e}rmino con dos literales.
    \item Cuatro cuadrados adyacentes representan cuatro minit\'{e}rminos $\rightarrow$ t\'{e}rmino con un literal.
    \item Ocho cuadrados adyacentes representan ocho minit\'{e}rminos $\rightarrow$ t\'{e}rmino constante 1.
\end{itemize}

\subsubsection*{3.1.3 Mapas de cuatro variables}
El mapa para funciones booleanas de cuatro variables presentan 16 minit\'{e}rminos,
por tanto el mapa consta de 16 cuadrados. El t\'{e}rmino correspondiente para cada
cuadrado de obtiene de la concatenaci\'{o}n del n\'{u}mero de fila con el n\'{u}mero
de la columna. Por ejemplo, los n\'{u}meros de la tercera fila (11) y la segunda
columna (01) al concatenarse dan el n\'{u}mero binario 1101, que es equivalente
binario al 13 decimal, representando al minit\'{e}rmino $m_{13}$.

Para el proceso de simplificaci\'{o}n podemos tener en cuenta:
\begin{itemize}
    \item Un cuadrado representa un minit\'{e}rmino $\rightarrow$ t\'{e}rmino con cuatro literales.
    \item Dos cuadrados adyacentes representan dos minit\'{e}rminos $\rightarrow$ t\'{e}rmino con tres literales.
    \item Cuatro cuadrados adyacentes representan cuatro minit\'{e}rminos $\rightarrow$ t\'{e}rmino con dos literales.
    \item Ocho cuadrados adyacentes representan ocho minit\'{e}rminos $\rightarrow$ t\'{e}rmino con una literal.
    \item Dieciseis cuadrados adyacentes representan dieciseis minit\'{e}rminos $\rightarrow$ constante 1.
\end{itemize}

\subsubsection*{3.1.4 Mapas de cinco variables}
El uso de mapas de m\'{a}s de cuatro variables no es tan simple. Un mapa de cinco variables ocupa 32 cuadrados,
y uno de seis variables ocupa 64 cuadrados. Por lo cual se complica progresivamente.
La mejor forma de visualizar estos mapas es imaginar que los mapas est\'{a}n uno encima del otro.
Cualesquier dos curadrados que queden uno encima del otro se consideran adyacentes.
Una alternativa es utilizar programas de computadora escritos espec\'{i}ficamente para
facilitar la simplificacion de funciones booleanas de cinco o m\'{a}s variables.

Se puede demostrar que cualesquier $2^k$ cuadrados adyacentes, para $k = (0, 1,2, ..., n)$,
en un mapa de $n$ variables, representan un \'{a}rea que produce un t\'{e}rmino de $n - k$
literales, con $n > k$, si $n = k$, el \'{a}rea representa una constante 1.

\subsection*{3.2 Implicantes primos}
Al escoger los cuadrados adyacentes hay que asegurarse de que se cubran todos los minit\'{e}rminos de la funci\'{o}n.
Evitar cubrir t\'{e}rminos redundantes cuyos t\'{e}rminos ya est\'{a}n cubiertos
por otros. Este procedimiento se podr\'{i}a hacer de manera sistem\'{a}tica con
el t\'{e}rmino de \textbf{implicante primo e implicante primo esencial}.

Un implicante primo es un t\'{e}rmino producto que se obtiene combinando el
n\'{u}mero m\'{a}ximo posible de cuadrados adyacentes en el mapa. Si un
minit\'{e}rmino de un cuadrado est\'{a} cubierto s\'{o}lo por un implicante
primo, se dice que es un \textbf{implicante primo esencial}.

\subsection*{3.3 Condiciones de indiferencia}
En la pr\'{a}ctica hay algunas aplicaciones en las que la funci\'{o}n no est\'{a} definida para ciertas
combinaciones de las variables. Las funciones con salidas no especificadas para
ciertas combinaciones de entradas se llaman \textbf{funciones incompletamente especificadas}.
Conviene usar estas condiciones de indiferencia en el mapa para simplificar a\'{u}n m\'{a}s la
expresi\'{o}n booleana. Se puede representar de la siguiente manera:
\begin{center}
    $F(w, x, y, z) = \sum (1, 3, 7, 11, 15) + d(0, 2, 5)$
\end{center}
Estas condiciones de indiferencia se pueden marcar bien
con 0 o 1, dependiendo de la aplicaci\'{o}n.

\subsection*{3.4 Funciones NAND y NOR}
Muchos circuitos digitales digitales se construyen con compuertas NAND y NOR en
lugar de compuertas AND y OR. Ya que son m\'{a}s f\'{a}ciles de fabricar con componentes
electr\'{o}nicos. Se dice que la compuerta NAND es una compuerta universal porque cualquier
sistema digital puede implementarse con ella. La compuerta NOR es otra compuerta universal. Se puede
demostrar que cualquier funci\'{o}n booleana se puede implementar con compuertas NOR.

\subsection*{3.5 Funci\'{o}n XOR}
La funci\'{o}n OR exclusivo (XOR), denotada por el s\'{i}mbolo $\oplus$, es una operaci\'{o}n
l\'{o}gica que efect\'{u}a la siguiente operaci\'{o}n:
\begin{center}
    $x \oplus y = xy' + x'y$
\end{center}
Es igual a 1 si s\'{o}lo $x$ es igual a 1 o s\'{o}lo $y$ es igual a 1, pero no si ambas son 1.
El NOR exclusivo, tambi\'{e}n llamado equivalencia, realiza la siguiente operaci\'{o}n:
\begin{center}
    $(x \oplus y)' = xy + x'y'$
\end{center}
Es igual a 1 si tanto x como y son 1 o si ambas son 0.

Se cumplen las identidades siguientes para la operaci\'{o}n XOR:
\begin{center}
    $x \oplus 0 = x$ \\
    $x \oplus 1 = x'$ \\
    $x \oplus x = 0$ \\
    $x \oplus x' = 1$ \\
    $x \oplus y' = x' \oplus y = (x \oplus y)' $
\end{center}

Tambi\'{e}n puede demostrarse que la operaci\'{o}n XOR es tanto conmutativa como asociativa;
es decir,
\begin{center}
    $x \oplus y = y \oplus x$ \\
    \begin{flushleft}
        y
    \end{flushleft}
    $x \oplus (y \oplus z) = (x \oplus y) \oplus z$
\end{center}

\subsubsection*{3.5.1 Funci\'{o}n impar}
La operaci\'{o}n XOR con tres o m\'{a}s variables se convierte en una funci\'{o}n booleana
ordinaria sustituyendo el s\'{i}mbolo $\oplus$ por su expresi\'{o}n booleana equivalente.
Por ejemplo:
\begin{align*}
    A \oplus B \oplus C & = (AB' + A'B)C' + (AB + A'B')C \\
                        & = AB'C' + A'BC' + ABC + A'B'C  \\
                        & = \sum (1, 2, 4, 7)
\end{align*}
La funci\'{o}n XOR de tres variables es una funci\'{o}n impar, es decir, es igual a 1 si el
n\'{u}mero de 1's en la funci\'{o}n es impar.

\subsubsection*{3.5.2 Generaci\'{o}n y verificaci\'{o}n de paridad}
Las funciones XOR son muy \'{u}tiles en los sistemas que requieren c\'{o}digos para
detectar y corregir errores. Un bit de paridad es un bit adicional que se incluye con
el mensaje binario de modo que el n\'{u}mero total de unos se impar o par. Se detecta
un error si la paridad recibida no corresponde con la transmitida. El transmior usar\'{i}a
un \textbf{circuito generador de paridad} y el receptor un \textbf{circuito verificador de paridad}.

\begin{table}[h]
    \centering
    \begin{tabular}{cccc}
        \multicolumn{3}{c}{Mensaje de tres bits} & Bit de Paridad             \\
        \cmidrule{1-3} \cmidrule{4-4} $x$        & $y$            & $z$ & $P$ \\
        \midrule 0                               & 0              & 0   & 0   \\ 0 & 0 & 1 & 1 \\ 0 & 1 & 0 & 1 \\ 0 & 1 & 1 & 0 \\
        1                                        & 0              & 0   & 1   \\
        1                                        & 0              & 1   & 0   \\
        1                                        & 1              & 0   & 0   \\
        1                                        & 1              & 1   & 1   \\
        \bottomrule
    \end{tabular}
    \caption{Tabla de verdad para de un generador deparidad par}
\end{table}