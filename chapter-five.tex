\newpage
\section*{5. L\'{o}gica secuencial sincr\'{o}nica}
\subsection*{5.1 Circutos secuenciales}
En la pr\'{a}tica la mayor\'{i}a de circuitos ocupan elementos de almacenamiento, por lo
que requieren que el sistema se describa en t\'{e}rminos de l\'{o}gica secuencial.
Consiste en circuito combinacional al que se conectan elementos de almacenamiento para 
formar una trayectoria de retroalimentaci\'{o}n. Estos elementos de almacenamiento son
\textbf{capaces de guardar informaci\'{o}n binaria}. La informaci\'{o}n almacenada en 
estos elementos en cualquier momento dado define el \textbf{estado} del circuito secuencial
en ese momento. 

El circuito secuencial recibe informaci\'{o}n binaria de entradas externas. \textbf{Estas entradas,
junto con el estado actual de los elementos de almacenamiento, determinan el valor binario de
las salidas.} Tambi\'{e}n determinan la condici\'{o}n para cambiar el estado de los elementos de
almacenamiento.
Las salidas de un circuito secuencial son funci\'{o}n no s\'{o}lo de las entradas, sino tambi\'{e}n
del estado actual de los elementos de almacenamiento. El siguiente estado de los elementos de almacenamiento
tambi\'{e}n es funci\'{o}n de entradas externas y del estado actual. As\'{i} un circuito secuencial
se especifica con una \textbf{sucesi\'{o}n temporal de entradas, salidas y estados.}

Existen dos tipos principales de circuitos secuenciales: \textbf{sincr\'{o}nicos} y \textbf{asincr\'{o}nicos}.

\subsubsection*{5.1.1 Circuitos secuenciales asincr\'{o}nicos}
El comportamiento de un circuito secuencial asincr\'{o}nico depende de las se\~{n}ales de entrada
en cualquier instante dado y del orden que cambian las entradas.
Los elementos de almacenamiento que suelen usarse son dispositivos de retardo de tiempo. La capacidad
de almacenamiento de un dispositivo de retardo de tiempo se debe al tiempo que la se\~{n}al tarda en 
propagarse por el dispositivo. En los sistemas asincr\'{o}nicos tipo compuerta, los elementos de almacenamiento
consisten en compuertas l\'{o}gicas cuyo retardo de propagaci\'{o}n hace posible el almacenamiento requerido.
As\'{i} un circuito secuencial asincr\'{o}nico podr\'{i}a considerarse como \textbf{un circuito combinacional
con retroalimentaci\'{o}n.} Gracias a esta retroalimentaci\'{o}n el circuito secuencial asincr\'{o}nico podr\'{i}a
volverse \textbf{inestable} ocasionalmente. Se estudiar\'{a} a mayor profundidad los circuitos secuenciales asincr\'{o}nicos
en el cap\'{i}tulo 9.

\subsubsection*{5.1.2 Circuitos secuenciales sincr\'{o}nicos}
Un circuito secuencial sincr\'{o}nico es un sistema cuyo comportamiento se define conociendo sus se\~{n}ales
en instantes discretos. La sincronizaci\'{o}n se logra con un dispositivo de temporizaci\'{o}n llamado \textbf{reloj}.
El cu\'{a}l genera un tren peri\'{o}dico de \textit{pulsos de reloj}.

\subsubsection*{5.1.3 Pulsos de reloj}

Estos pulsos de reloj se distribuyen por todo el sistema de modo que los elementos de almacenamiento se vean afectados
al llegar cada pulso. Casi nunca manifiestan problemas de estabilidad y es f\'{a}cil dividir su temporizaci\'{o}n
en pasos discretos independientes.

Los elementos de almacenamiento empleados en circutios secuenciales con reloj se llaman \textit{flip flops}.
Un flip flop es un dispositivo capaz de almacenar un bit de informaci\'{o}n. Estos reciben sus entradas de un 
circuito combinacional y tambi\'{e}n una se\~{n}al de reloj, cuyos pulsos se presentan a intervalos fijos de tiempo.
El estado de un flip flop cambia s\'{o}lo cuando llega un pulso de reloj. Por tanto, la transici\'{o}n de un estado
al siguiente se da \'{u}nicamente a intervalos de tiempo prestablecidos, dictados por los pulsos de reloj.

\subsection*{5.2 Latches}
\subsubsection*{5.2.1 Latch SR y Latch S'R'}
\subsubsection*{5.2.3 Latch D}

\subsection*{5.3 Flip-flops}
\subsubsection*{5.3.1 Flip-flop D}
\subsubsection*{5.3.2 Flip-flop JK}
\subsubsection*{5.3.3 Flip-flop T}